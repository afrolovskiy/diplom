\Introduction

Интенсивное развитие информационных технологий и расширение сферы их применения привело к значительному увеличению сложности используемого программного обеспечения, а также росту количества и критичности выполняемых им функций. С увеличением сложности возрастает количество ошибок. Ущерб от них несет существенные последствия. Одними из наиболее опасных являются ошибки, связанные с гонками при работе с данными. Они носят стохастический характер, что обуславливает сложность их выявления и исправления.

В данной работе будут рассмотрены методы статического поиска гонок, возникающих при множественном доступе к разделяемой памяти. 
Под \textbf{состоянием гонки} при множественном доступе к разделяемой памяти будем понимать ситуацию, когда два или более потоков одновремено совершают доступ к разделяемой области памяти, и по крайней мере хотя бы один из них выполняет операцию записи в неё.

В листинге~\ref{lst:race} показан пример программы, в которой возможно возникновение гонок при доступе к разделяемой переменной. Доступ к разделяемой переменной $count$ в функции $foo$ является не защищенным ни одним из средств взаимоисключения. Это может привести к возникновению гонок при одновременном доступе к ней из различных потоков.

В листинге~\ref{lst:protected} показан пример уже исправленной программы, в которой проблема гонок при множественном доступе к разделяемой памяти устранена посредством использования мьютекса.

\lstinputlisting[language=C,caption=Гонка при доступе к разделяемой переменной(\Code{race.c}),label=lst:race]{../src/race.c}

\lstinputlisting[language=C,caption=Защищённый доступ к разделяемой переменной(\Code{protected.c}),label=lst:protected]{../src/protected.c}
