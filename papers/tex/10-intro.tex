\Introduction

Интенсивное развитие информационных технологий и расширение сферы их применения привело к значительному увеличению сложности используемого программного обеспечения, а также росту количества и критичности выполняемых им функций. С увеличением сложности возрастает количество ошибок. Ущерб от них несет существенные последствия. Одними из наиболее опасных являются ошибки, связанные с гонками при работе с данными. Они носят стохастический характер, что обуславливает сложность их выявления и исправления.

Бла-Бла-Бла (про актуальност,классификацию ресурсов, гонок, типы создания  потоков, признаки возникновения гонок).

Под \textbf{состоянием гонки} при множественном доступе к разделяемой памяти будем понимать ситуацию, когда два или более потоков одновремено совершают доступ к разделяемой области памяти, и по крайней мере хотя бы один из них выполняет операцию записи в неё. 

\lstinputlisting[language=C,caption=Гонка при доступе к переменной(\Code{race.c}),label=lst:race]{../src/race.c}

\lstinputlisting[language=C,caption=Пример (\Code{protected.c}),label=lst:protected]{../src/protected.c}

В листинге~\ref{lst:race} показан пример программы ..

В листинге~\ref{lst:protected} показан пример программы ..
