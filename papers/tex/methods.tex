\chapter{Подходы к поиску гонок}
\label{cha:methods}

Основные подходами к поиску гонок в программах являются:
\begin{itemize}
\item формальная верификация,
\item динамический анализ,
\item статический анализ.
\end{itemize}

\textbf{Формальная верификация} заключается в в установлении соответствия между
программой и требованиями к программе, описывающими цель разработки
\cite{kropacheva-formal-verification}. Основными подходами являются
метод проверки моделей и дедкутивный анализ.

\textbf{Дедуктивный анализ} направлен на изучение логики преобразования данных,
определяемой решением поставленной задачи. Основная идея метода заключается
в том, что последовательными преобразованиями программа переводится в формулу
логики, а тербования к программе либо изначально формулируются на языке
логики, либо переводятся с какого-либо языка на язык логики. После этого доказательство корректности программы сводится к доказательству
эквивалентности двух формул, что осуществляется с помощью методов,
разработанных в логике. Данный метод хорошо разработан для последовательных
программ, для параллельных - процесс сильно усложняется
\cite{kropacheva-formal-verification}.

\textbf{Метод проверки моделей} заключается в том, желаемые свойства поведения
реагирующей системы проверяются на заданной системе (модели) путем
исчерпывающего перебора всех состояний, достижимых системой, и всех поведений,
проходящих через эти состояния \cite{klark-model-checking}. Основным недостатком
данного метода является <<комбинаторный взрыв>> в пространстве состояний,
возникающий в случае, когда исследуемая система состоит из многих компонент,
переходы в которых выполняются параллельно.

\textbf{Динамический анализ} основан на изучении потока событий, генерируемых
программой во время выполнения \cite{kovega-dynamic-analysis}. Недостатком
данного метода является то, что состояние гонки может быть зафиксировано
только лишь в том случае, если оно возникло при прогоне очередного теста,
а, следовательно, вероятность нахождения состояния гонки сильно зависит от
степени покрытия тестами подобного рода ситуаций.

\textbf{Статический анализ} основан на анализе исходного кода программы.
Достоинством данного метода является теоретическая возможность анализа
всех возможных путей выполнения программы. Недостатком является наличие ложных
срабатываний, то есть обнаружение ситуаций гонок в тех местах программы,
где их нет, что усложняет анализ и выявлениетех результатов, которые
соответствуют действительным ситуациям гонок. Примером такой ситуации
является инициализация переменных в момент, когда программы выполняется
в рамках одного процесса или потока.
