\chapter{Подходы к поиску гонок}
\label{cha:methods}

Основными подходами к поиску гонок в программах являются:
\begin{itemize}
\item формальная верификация,
\item динамический анализ,
\item статический анализ.
\end{itemize}

\textbf{Формальная верификация} основана на установлении соответствия между программой и требованиями к программе, описывающими цель разработки \cite{kropacheva-formal-verification}. Основными методами формальной верификации являются метод проверки моделей и дедуктивный анализ.

Основная идея \textbf{дедуктивного анализа} состоит в том, чтобы последовательными преобразованиями привести программу в формулу логики (требования к программе либо изначально формулируются на языке логики, либо переводятся с какого-либо языка на язык логики). После этого доказательство корректности программы сводится к доказательству эквивалентности двух формул, что осуществляется с помощью методов, разработанных в логике. Данный метод хорошо разработан для последовательных программ, для параллельных - процесс сильно усложняется \cite{kropacheva-formal-verification}.

\textbf{Метод проверки моделей} заключается в том, желаемые свойства поведения реагирующей системы проверяются на заданной системе (модели) путём исчерпывающего перебора всех состояний, достижимых системой, и всех поведений (путей), проходящих через эти состояния \cite{klark-model-checking}. Основным недостатком данного метода является <<комбинаторный взрыв>> в пространстве состояний, возникающий в случае, когда исследуемая система состоит из многих компонент, переходы в которых выполняются параллельно.

\textbf{Динамический анализ} основан на изучении потока событий, генерируемых программой во время выполнения \cite{kovega-dynamic-analysis}. Недостатком данного метода является то, что состояние гонки может быть зафиксировано, только если оно возникло в проверяемом варианте исполнения программы, а, значит, нет гарантии, что оно не может возникнуть в каком-то ином варианте. Другим существенным недостатком является то, что большинство средств динамического анализа зависит от оснащения приложения средствами мониторинга, что может менять поведение исполняющей среды.

\textbf{Статический анализ} основан на анализе исходного кода программы. Достоинством данного метода является теоретическая возможность анализа всех возможных путей выполнения программы.  Недостатком является наличие ложных срабатываний, то есть обнаружение ситуаций гонок в тех местах программы, где их нет, что усложняет анализ и выявление тех результатов, которые соответствуют действительным ситуациям гонок. Примером такой ситуации является инициализация переменных в момент, когда программы выполняется в рамках одного процесса или потока.

Далее будут рассматриваться только методы поиска гонок, основанные на статическом анализе.
