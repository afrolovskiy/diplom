\Conclusion

Контекстная зависимость является не единственной проблемой, возникающей при статическом поиске гонок в программах на языке C. Помимо системных мьютексов существуют и другие способы обеспечения монопольного доступа к данным, например, семафоры. Несмотря на то, что поведение семафоров может быть проэмулировано с использованием мьютексов, это может привести к возникновению новых ложных срабатываний. Также при анализе указателей возникают проблемы, связанные с динамическим выделением памяти. И наконец, разрешение проблем, связанных с условными блокировками, также явлется весьма нетривиальной задачей анализа.

Самым надёжным способом обеспечения отсутствия гонок является исполнение только одного потока. Даже в многопоточной программе поток может не всегда выполняться параллельно с другими потоками. Существует много механизмов обеспечения синхронизации без блокировок (англ. lock-free). Например, предположим, что есть главный поток и $k$ рабочих потоков (англ. workers). Главный поток содержит массив $A$ с $k$ элементами (по одному для каждого из потоков) такой, что $A[i]$ влияет на $i$ поток. Более того, предположим, что основной поток инициализирует этот массив перед порождением рабочих потоков (англ. worker threads) и обрабатывает этот массив после завершения всех потоков. Хотя нет блокировок, программа не содержит гонок, т.к. главный поток осуществляет доступ к массиву только тогда, когда рабочие потоки не имеют, и рабочие потоки следуют этому соглашению, что обеспечивает монопольный доступ.

Обычно выделяют временные фазы работы программы такие, как инициализация, обработка и постобработка. Анализатор, имея информацию о том, к каким ресурсам в какие моменты времени производится доступ, должен уметь определять, какие из выполняющихся потоков действительно конфликтуют. Традиционный подход заключается в попытке частично упорядочить инструкции по последовательности исполнения, когда это возможно. Гонка может произойти при одновременном доступе только в том случае, если нет ограничений последовательности выполнения.
