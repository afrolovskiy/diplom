\Introduction

Интенсивное развитие информационных технологий и расширение сферы их применения привело к значительному увеличению сложности используемого программного обеспечения, а также к росту количества и критичности выполняемых им функций. С увеличением сложности возрастает количество ошибок. Ущерб от них несет существенные последствия. Одними из наиболее опасных являются ошибки, связанные с гонками при работе с данными, хранящимися в памяти. Они носят стохастический характер, что обуславливает сложность их выявления и исправления.

Существует три основные группы методов поиска гонок в программах: формальная верификация, динамические и статические методы. Каждая из групп имеет свою область применения. Программное обеспечение, реализующие статический поиск гонок, является удобным инструментом, позволяющим преждевременно еще на этапе разработки выявлять и устранять дефекты программ, связанные с возможными возникновениями гонок при доступе к разделяемым ресурсам.

Язык Си в настоящее время играет важную роль в области промышленной разработки программного обеспечения, поэтому целью работы является разработка статического метода поиска гонок в программах на языке Си. Для достижения поставленной цели необходимо решить следующие задачи:

\begin{itemize}
\item выполнить анализ методов поиска гонок в программах, выявить их достоинства и недостатки;
\item разработать метод статического поиска гонок при доступе к разделяемой памяти;
\item разработать алгоритмы, входящие в состав предложенного метода;
\item разработать ПО, реализующее предложенный метод;
\item провести исследование разработанного метода.
\end{itemize}

Данная пояснительная записка имеет следующую структуру:

\begin{itemize}
  \item В главе 1 рассмотрены существующие методы поиска гонок при доступе к разделяемой памяти.
  \item В главе 2 приведено описание разработанного метода поиска гонок в программах на языке Си и используемых алгоритмов.
  \item В главе 3 рассмотрено проектирование программного обеспечения, реализующего разработанный метод.
  \item В главе 4 приведено описание экспериментов, проводимых с разработанным программным обеспечением, и их результаты.
\end{itemize}
