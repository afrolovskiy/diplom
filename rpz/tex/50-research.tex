\chapter{Исследовательский раздел}
\label{cha:research}

В данном разделе описаны эксперименты, проводимые с разработанным программным обеспечением. Эксперименты проводятся с целью определения скоростных характеристик разработанного программного обеспечения и метода в целом, производится оценка точности обнаружения ситуаций гонок в программах в зависимости от их структуры.

\section{Условия проведения экспериментов}

Проведение экспериментов производилось в следующих условиях:
\begin{enumerate}
    \item аппаратное обеспечение:
        \begin{enumerate}
            \item AMD Turion(tm) X2 Ultra Dual-Core Mobile ZM-82 2,2GHz;
            \item 3096 Мб ОЗУ;
        \end{enumerate}
    \item программное обеспечение:
        \begin{enumerate}
            \item ОС Fedora GNU/Linux 20.0 3.11.10-301.fc20.i686;
            \item GCC 4.8.2;
            \item Python 2.7.5;
            \item gcc-python-plugin 0.12.
        \end{enumerate}
\end{enumerate}

\section{Исследование скоростных характеристик}

Для проведения экспериментов по определению скоростных характеристик за основу была взята задача <<читатели-писатели>>. В ней предполагается, что есть некоторый общий для всех потоков ресурс. Часть потоков получает к нему доступ только для чтения, а часть - для записи. При этом чтение может осуществляться одновременно из нескольких потоков. Код анализируемой программы представлен в листинге ~\ref{lst:readers-writers}. 

\lstinputlisting[language=C,caption=Код решения задачи "читатели-писатели" (\Code{readers-writers.c}),label=lst:readers-writers]{inc/src/readers-writers.c}

В процессе проведения экспериментов проводилось изменение числа потоков, создаваемых в функции \texnbf{main}, и ограничения на количества раз, которое каждый базовый блок может встретиться в анализируемом пути выполнения функции. Графики зависимостей количества анализируемых путей и количества анализируемых инструкций от изменения значения ограничения, накладываемого на количество раз, которое базовый блок может встретиться в пути представлены на рис. и на рис. соответственно. Зависимости времени анализа от количества анализируемых путей, количества потоков и путей представлены на рис., рис. и рис. соответственно. Видно, что

\section{Исследование точности}

Бла-бла
