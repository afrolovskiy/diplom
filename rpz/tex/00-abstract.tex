\begin{abstract}
Ква­лификационная работа магистра по направлению <<Информатика и вычислительная техника>> на тему <<Метод статического поиска гонок в программах на языке Си>> посвящена разработке метода статического поиска гонок на основе относительного множества блокировок.

В ходе работы был проведен анализ предметной области, рассмотрены существующие подходы к поиску гонок, выявлены их достоинства и недостатки. На основе проведенного анализа был сделан выбор в сторону разработки статического метода.

В расчётно-пояснительной записке к данной работе представлено подробное описание алгоритмов, используемых на каждом из этапов разботанного метода. В ходе работы было разработано программное обеспечение на языке Python с использованием библиотеки gcc-python-plugin, реализующее предложенный метод. С использованием разработанного программного обеспечения было проведено исследование времени анализа программы в зависимости от количества анализируемых путей, инструкций и количества потоков на основе задачи <<читатели-писатели>>. Результаты показали, что время анализа линейно зависит от количества анализируемых путей и инструкций. Также было установлено, что количество анализируемых потоков слабо влияет на время анализа. Это объясняется тем, что анализ каждой функции программы выполняется только один раз, и в местах вызова функции применяются результаты анализа, полученные ранее.
\end{abstract}
