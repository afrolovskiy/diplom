\Conclusion

В результате проделанной дипломной работы был проведен анализ существующих методов поиска гонок, выявлены их достоинства и недостатки. На основе анализа существующих методов статического поиска гонок был разработан метод статического поиска гонок, основанный на использовании относительных множеств блокировок, были разработаны и реализованы алгоритмы, входящие в состав разработанного метода.

На основе предложенного метода был разработан статический анализатор кода, реализованный в виде загружаемого модуля к компилятору gcc. С его использованием было проведено исследование скоростных характеристик предложенного метода. Было получено, что с ростом значения ограничения, накладываемого на максимальное количество вхождений базового блока в путь, происходит экспоненциальный рост количества анализируемых путей и инструкций. Также было выявлено, что время анализа  зависит линейно от количества анализируемых путей и инструкций. Помимо этого было обнаружено, что количество анализируемых  потоков слабо влияет на время анализа. Это объясняется тем, что анализ каждой функции программы программы выполняется только один раз, и в местах вызова функции применяются результаты анализа, полученные ранее.

В качестве возможного улучшения предложенного метода можно добавить возможность определения параллельно выполняющихся потоков. Это можно сделать, например, при помощи введения множества активных потоков, состоящиего из множества возможно выполняющихся потоков и множества остановленных потоков. Тогда в защищенный доступ добавится еще один элемент, и при сравнении доступов на наличие гонок надо будет дополнительно проверять пересечение множеств возможно выполняющихся потоков, и в случае, когда оно не является пустым производить сравнение достуов. Это позволит избежать проблем, связанных с предположением о параллельном исполнении всех потоков программы.
